\section{Numerical Model}
The full Julia code used to execute this method used in this derivation can be found in \appref{A}. In order to solve this system numerically we must bring this system to a first-order differential equation.
\subsection{Isolating \(\ddot\theta\)}

We start from our derived second-order equations from last section \eqref{theta1}, and \eqref{theta2}.

Rearrange each equation to isolate the terms that multiply the accelerations on the left and move the remaining terms to the right-hand side:
\begin{align}
  (m_1 + m_2) l_1 \ddot{\theta}_1
  \+\ m_2 l_2 \cos(\theta_1-\theta_2)\ddot{\theta}_2
  &=\ -m_2 l_2 \dot{\theta}_2^2 \sin(\theta_1-\theta_2) - (m_1+m_2) g \sin\theta_1, \label{eq:rearr1}\\
  m_2 l_1 \cos(\theta_1-\theta_2)\ddot{\theta}_1
  \+\ m_2 l_2 \ddot{\theta}_2
  &=\ -\big( - m_2 l_1 \dot{\theta}_1^2 \sin(\theta_1-\theta_2) + m_2 g \sin\theta_2\big). \label{eq:rearr2}
\end{align}

Define the mass matrix \(M(\theta)\) and the right-hand side vector \(b(\theta,\omega)\) (with \(\omega_i=\dot\theta_i\)):
\[
M(\theta)=
\begin{pmatrix}
a_{11} & a_{12} \\[6pt]
a_{21} & a_{22}
\end{pmatrix}
=
\begin{pmatrix}
(m_1+m_2) l_1 & m_2 l_2 \cos(\theta_1-\theta_2) \\[6pt]
m_2 l_1 \cos(\theta_1-\theta_2) & m_2 l_2
\end{pmatrix},
\]
\[
b(\theta,\omega)=
\begin{pmatrix}
b_1\\[6pt] b_2
\end{pmatrix}
=
\begin{pmatrix}
-m_2 l_2 \omega_2^2 \sin(\theta_1-\theta_2) - (m_1+m_2) g \sin\theta_1 \\[6pt]
\m_2 l_1 \omega_1^2 \sin(\theta_1-\theta_2) - m_2 g \sin\theta_2
\end{pmatrix}.
\]

Thus the coupled accelerations satisfy the matrix equation
\[
M(\theta)\ddot{\theta} =\ b(\theta,\omega),
\qquad
\ddot\theta=\begin{pmatrix}\ddot\theta_1\\[4pt]\ddot\theta_2\end{pmatrix}.
\]

For a \(2\times2\) matrix we can solve systems by computing the inverse by hand so we can easily find \(\ddot\theta\).
Apply \(M^{-1}\) to \(b\):
\[
\ddot\theta = M^{-1} b
= \frac{1}{\det M}
\begin{pmatrix}
a_{22} & -a_{12} \\[6pt]
- a_{21} & a_{11}
\end{pmatrix}
\begin{pmatrix} b_1 \\[6pt] b_2 \end{pmatrix}.
\]
Therefore the components are
\begin{align}
\ddot{\theta}_1
&= \frac{1}{\det M}\Big( a_{22}b_1 - a_{12}b_2 \Big), \label{eq:ddot1}\\[6pt]
\ddot{\theta}_2
&= \frac{1}{\det M}\Big( -a_{21}b_1 + a_{11}b_2 \Big). \label{eq:ddot2}
\end{align}
Substituting \(a_{ij}\) and \(b_k\) explicitly gives:
\begin{equation}\label{ddtheta1}
\begin{split}
\ddot{\theta}_1
= \dfrac{1}{\det M}\Big(
  m_2 l_2\big(-m_2 l_2 \omega_2^2 \sin\Delta &- (m_1+m_2) g \sin\theta_1\big)\\
  &- m_2 l_2 \cos\Delta\big(m_2 l_1 \omega_1^2 \sin\Delta - m_2 g \sin\theta_2\big)
\Big)
\end{split}
\end{equation}
and

\begin{equation}\label{ddtheta2}
\begin{split}
\ddot{\theta}_2
= \dfrac{1}{\det M}\Big(
  - m_2 l_1 \cos\Delta\big(&-m_2 l_2\omega_2^2 \sin\Delta - (m_1+m_2) g \sin\theta_1\big)\\
&+ (m_1+m_2)l_1\big(m_2 l_1 \omega_1^2 \sin\Delta - m_2 g \sin\theta_2\big)
\Big).
\end{split}
\end{equation}

\subsection{Convert to a first-order system.}
Introduce the state vector
\[
X = \begin{pmatrix} \theta_1 \\[4pt] \theta_2 \\[4pt] \omega_1 \\[4pt] \omega_2 \end{pmatrix},
\qquad \omega_i=\dot\theta_i.
\]
The first-order system is
\[
\dot X = F(X)
= \begin{pmatrix}
\omega_1 \\[6pt]
\omega_2 \\[6pt]
\ddot\theta_1(\theta_1,\theta_2,\omega_1,\omega_2) \\[6pt]
\ddot\theta_2(\theta_1,\theta_2,\omega_1,\omega_2)
\end{pmatrix},
\]
where \(\ddot\theta_1,\ddot\theta_2\) are given by equations \eqref{eq:ddot1} and \eqref{eq:ddot2} respectively.
