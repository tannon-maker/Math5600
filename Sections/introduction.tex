\section{Introduction}
A single pendulum is a classical example of simple harmonic motion. When constrained to small angles the pendulum will swing periodically and consistently. They are so predictable that some have been used to keep time. By simply adding a second pendulum at the end of the first, the system transforms into a classical example of chaotic motion. Even though these two systems are governed by the same physical laws of motion and only being acted upon by one force (gravity), a double pendulum is \emph{heavily} dependent on initial conditions. We will use an approximation to solve this equation to view the behavior; the approximation we will use is the fourth order Runge-Kutta method.
